\section{Testing}
During the development of the GeoLogger multiple test where performed. The most important tests will be listed in the following sections along with the data obtained from the tests. 

\subsection{Wixel range test}
The test was performed in the Reykjavík University electronics lab. The room is 18 meters across and marked lines with 2.5 meters  between them were laid down. A Wixel connected to a serial monitor was placed at one end of the room and another Wixel was moved between the lines regularly sending data strings. The table below shows if the data was received on the serial monitor. Each test was performed two times.
\begin{table}[H]
		\centering
		\label{tab:Table_2}
		\caption{Wixel range test}
 	 	     \begin{tabular}
 	 	     	{| p{2cm} |  p{2cm} | p{3cm} | p{2cm}  | p{2cm} | p{3cm} |}
 	 	    	 \hline
 	 	    	  Number & Distance	& Received & Number	& Distance   & Received 	\\
 	 	    	  	 \hline
 	 	  		 1 & 	2.5m & yes	& 8	 & 2.5m & yes\\
 	 	  		   \hline
 	 	  		 2 &	5m	 & yes	& 9	 &   5m & yes\\
 	 	  		   \hline 
 	 	  	 	 3 &	7.5m & yes	& 10 & 7.5m & yes\\
 	 	  	 	   \hline 
 	 	  	 	 4 &    10m	 & yes  & 11 &	10m	& yes\\
 	 	  	 	   \hline 
 	 	  	 	 5 &	12.5m& yes	& 12 & 12.5m& yes\\
 	 	  	 	   \hline 
 	 	  	 	 6 &	15m	 & yes	& 13 &  15m & yes\\
 	 	  	 	   \hline 
 	 	  	 	 7 &	18m	 & no	& 14 &  18m	& no \\
 	 	  	 	   \hline 
 
 	 	     \end{tabular}\\
 	 	    
 	 	 \end{table}
 	 	 
\subsection{Lab test one}
Lab test one was performed in the Reykjavík University electronics lab. The mother hub was placed at the centre of the room and the wireless sensor modules were placed in three corners of the room. The mother hub was programmed to poll the wireless sensor modules for data every five minutes and then send the gather data to the HTTP server every 15 minutes. The aim of the test was to see if the system was ready for field testing and to detect errors. No errors were detected during the test and the GeoLog was made ready for field testing. Table~\ref{tab:Table_3} shows the results of lab test one.

\begin{table}[H]
		\label{tab:Table_3}
		\caption{Lab test 1}
	\begin{tabular}
		{| p{6cm} |  p{4cm} | p{2cm} | p{3cm}  |}
		\hline
		Transmit time & Timestamp [sec] & Sensor ID & Temperature $\degree$C  \\ \hline
		05. November 2014 15:24 & 51   & 1 & 23 \\  \hline
		05. November 2014 15:24 & 51   & 2 & 24 \\  \hline
		05. November 2014 15:24 & 51   & 3 & 21 \\  \hline
		05. November 2014 15:24 & 56   & 1 & 23 \\  \hline
	    05. November 2014 15:24 & 57   & 2 & 24 \\  \hline
		05. November 2014 15:24 & 57   & 3 & 21 \\  \hline
		05. November 2014 15:24 & 58   & 1 & 23 \\  \hline
		05. November 2014 15:24 & 58   & 2 & 24 \\  \hline
		05. November 2014 15:24 & 59   & 3 & 21 \\  \hline
		05. November 2014 15:58 & 137  & 1 & 23 \\  \hline
		05. November 2014 15:58 & 137  & 2 & 24 \\  \hline
		05. November 2014 15:58 & 137  & 3 & 20 \\  \hline
		05. November 2014 16:15 & 438  & 1 & 23 \\  \hline
		05. November 2014 16:15 & 438  & 2 & 24 \\  \hline
		05. November 2014 16:15 & 438  & 3 & 20 \\  \hline
		05. November 2014 16:15 & 740  & 1 & 23 \\  \hline
		05. November 2014 16:15 & 740  & 2 & 24 \\  \hline
		05. November 2014 16:15 & 741  & 3 & 20 \\  \hline
		05. November 2014 16:15 & 1041 & 1 & 23 \\  \hline
		05. November 2014 16:15 & 1041 & 2 & 25 \\  \hline
		05. November 2014 16:15 & 1041 & 3 & 21 \\  \hline
	
		
		
	\end{tabular}
	
\end{table}

\subsection{Field test one}
Field test one was performed in Öskuhlið next to Reykjavík University. The mother hub was attached to a tree along with a battery pack containing 5 parallel connected 9V battery's. The wireless sensor modules were also attached to trees with 15 meters radius around the mother hub. The mother hub was programmed to gather data from the wireless sensor modules with five minutes intervals and transmit gathered data every 15 minutes. The idea was to let the GeoLog run while transmitting data in order to monitor problems and to measure power consumption. As can be seen on the time stamp  in table~\ref{tbl:Table_4} the system was running when it was placed in the field, this was because the system was started up in the lab to confirm that it was going to send data before it was placed in the field. \\
Table~\ref{tbl:Table_4} shows that the mother station was able to send two whole data packs before stopping. When no data was received from the GeoLog for two hours a decision was made to restart the system and manually send data. As can be seen on the time stamp in table~\ref{tbl:Table_4} included in the last data pack that was sent are measurements that where taken before the system stooped. This shows that the function that stores data on the EEPROM is doing what it designed to do, that is to make sure that even though the system stops and is not able to transmit data no information is lost. \\
The likely reason why the system stopped sending data is the mother hub did not receive data from one or more of the wireless sensor modules. More research has to be done to confirm this.
	\begin{table}[H]
		
			\caption{Field test 1}
				\label{tbl:Table_4}
		\begin{tabular}
			{| p{6cm} |  p{4cm} | p{2cm} | p{3cm}  |}
			\hline
			Transmit time & Timestamp [sec] & Tensor ID & Temperature $\degree$C  \\ \hline
			05. November 2014 17:13 & 2295 & 1 & 0   \\  \hline
			05. November 2014 17:13 & 2298 & 2 & 1   \\  \hline
			05. November 2014 17:13 & 2299 & 3 & 3   \\  \hline
			05. November 2014 17:13 & 2619 & 1 & -5  \\  \hline
			05. November 2014 17:13 & 2619 & 2 & -3  \\  \hline
			05. November 2014 17:13 & 2620 & 3 & -1  \\  \hline
			05. November 2014 17:13 & 2924 & 1 & -7  \\  \hline
			05. November 2014 17:13 & 2925 & 2 & -6  \\  \hline
			05. November 2014 17:13 & 2925 & 3 & -5  \\  \hline
			05. November 2014 17:31 & 3234 & 1 & -8  \\  \hline
			05. November 2014 17:31 & 3235 & 2 & -8  \\  \hline
			05. November 2014 17:31 & 3235 & 3 & -6  \\  \hline
			05. November 2014 17:31 & 3558 & 1 & -10 \\  \hline
			05. November 2014 17:31 & 3559 & 2 & -8  \\  \hline
			05. November 2014 17:31 & 3560 & 3 & -8  \\  \hline
			05. November 2014 17:31 & 3892 & 1 & -10 \\  \hline
			05. November 2014 17:31 & 3893 & 2 & -9  \\  \hline
			05. November 2014 17:31 & 3893 & 3 & -9  \\  \hline
			05. November 2014 19:35 & 4198 & 1 & -10 \\  \hline
			05. November 2014 19:35 & 4212 & 2 & -9  \\  \hline
			05. November 2014 19:35 & 4212 & 3 & -9  \\  \hline
			05. November 2014 19:35 & 4521 & 1 & -11 \\  \hline
			05. November 2014 19:35 & 4526 & 2 & -10 \\  \hline
			05. November 2014 19:35 & 4527 & 3 & -10 \\  \hline
			05. November 2014 19:35 & 12   & 1 & -14 \\  \hline
			05. November 2014 19:35 & 15   & 2 & -12 \\  \hline
			05. November 2014 19:35 & 15   & 3 & -12 \\  \hline

		\end{tabular}
		
	\end{table}
	
\subsection{Regular user tests}
Tests that the user needs to perform regularly are: The user needs to calibrate any sensors he wants to implement with the system, he needs to do ranges tests for different wireless sensor modules and for the different conditions that he plans on using them inn. The user also needs to do some tests on the GSM connection on the feeld where he plans on putting up the mother hub so that he can be sure that the data will be delivered.

