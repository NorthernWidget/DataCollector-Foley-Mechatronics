\documentclass{beamer}
% % Example from http://www.math.umbc.edu/~rouben/beamer/quickstart-Z-H-3.html#node_sec_3
%\usepackage{default}
\usetheme{default}
\usepackage[utf8]{inputenc}  % this allows latex to understand UTF
                             % encoding in the file. This is
                             % important for Icelandic characters.
\usepackage[T1]{fontenc} % this enables icelandic characters in the output

% % Use the UMBC theme
% Template derived from hfc.tex by
% Rouben Rostamian <rostamian@umbc.edu>
% August 31, 2004 
\usetheme{ru1}
\useinnertheme{ruboxes}
\setbeamercolor{ruboxes}{bg=violet!12,fg=black}

\usepackage{rotating} % for defining \schwa
\newcommand{\schwa}{\raisebox{1ex}{\begin{turn}{180}e\end{turn}}}

\title{GeoLog}
\subtitle{Milestone presentation}
\author[P. Helgasson, S. Ólafsson, S. Magnússon, \& Þ. Tómasarson]{Páll Helgason pallsh12@ru.is, Sindri Ólafsson sindrio12@ru.is, Sveinn Elmar Magnússon sveinnm12@ru.is, \& Þór Tómasarson thortom12@ru.is}
\institute[RU]{
  Department of Science and Engineering (TVD) \\
  Reykjavík University \\
}
\date{October 28, 2014} %% Put the real presentation day so it doesn't
                        %% change later
\begin{document}

%----------- titlepage ----------------------------------------------%
\begin{frame}[plain]
  \titlepage
\end{frame}

%----------- slides ----------------------------------------------%
\begin{frame}{Presentation Hints}
Don't include these in your presentation, they are just here to help you.
\begin{itemize}
\item More pictures, less text.
\item Scan and look around the audience.
\item More slides, less time per slide.
\item Practice until you can do the slides without looking at them.
  If you must have notes, put them on cards.  Do not read from a page
  nor the slides.  It looks bad.
\item 45 -- 60 seconds per slide, no more.
\item Tell a story (make sure it flows).
\end{itemize}
\end{frame}

\begin{frame}
\frametitle{Overview of the talk}
\begin{itemize}
\item Introduction
\item Use case
\item Design
\item Status
\item Tasks Remaining
\end{itemize}
\end{frame}

\begin{frame}{Introduction}
\begin{itemize}
\item What is this project?
\item Why is it called that?  
\item What is the problem it is trying to solve?
\item Who cares?
\end{itemize}
\end{frame}

\begin{frame}{Use case}
\begin{itemize}
\item How do you use it?  (Give examples and walkthrough)
\end{itemize}
\end{frame}

\begin{frame}{Design: Overview}
What are the pieces of your design (software, hardware, etc.)  and how
do they work together?  Put some pictures here to help explain what is
going on.  Remember to cite all pictures and text using BibTeX.  If you cited
the textbook\cite{carryer2011IntroMechatronics}, that is what it would
look like.
\end{frame}

\begin{frame}{Design: Bill of materials (BOM)}
What are the electronics and materials?  What do they do? 

\begin{itemize}
\item 1 x Arduino Mega (Model: 2560 R3 from Sparkfun) \$45.95
\item 1 x GSM Module with shield  (Model: SM5100B from Sparkfun) \$99.95
\item 3 x Wixel   (With CC2511F32 microc. from Sparkfun) \$19.95
\item 3 x Sensors
\end{itemize}
\end{frame}

\begin{frame}{Status}
What have you done?  What is the current state?  Is there any interesting/hard problems
that need to be solved?  How are you going to solve them?
\end{frame}

\begin{frame}{Tasks remaining}
What is left to be done?  This should be broken down by the time remaining in class.
\end{frame}

\begin{frame}{References}
Thank you for your time.
Questions?
\bibliographystyle{IEEEtran}
\bibliography{references}
\end{frame}

\end{document}
